\begin{abstract}
%In this paper w
We present a method to extract models from reference animations that can reproduce movement from user input. 
%Although 
Much work has been to done to create realistic full body character animation there is a lack of methods for performing the supplementary task of accurate modeling between user input and future trajectory. This is important for systems that use %\kenny{does reviewer know what this is?} 
weak control signals to guide animation synthesis
%, but e
Especially for computer games where the character position is typically mapped directly to the estimated trajectory with animations overlaid as a secondary effect that is not allowed to diverge. \changed{When a visual discrepancy artifacts appear current practice requires a tedious manual alignment process to avoid this. Our method provides an automated alternative to current practice.} \changed{Input control is ill-defined as the user may not be able to faithfully reconstruct the full movement details using a gamepad. Hence, we regularize the task by introducing control genomes to represent reduced user control signals, thereby deferring complexity to our models.} We further introduce movement models using a modular approach where primitives are combined and we demonstrate a model for plane locomotion in games. Finally the modeling task is further addressed by animation alignment and trajectory estimation where unwanted details are filtered.
\end{abstract}
