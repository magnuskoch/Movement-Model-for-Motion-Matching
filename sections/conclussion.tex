\section{Discussion and Limitations}
We have presented a method to 

We have demonstrzte thjat our model can fit to local segment, but a global fitting remains as a challenge.

heterogenous animation. Actor can choose to turn more abrubtly. We can fix some witha lignment, but this makes movementm unnatural. We can add more to conotrl genome but this conflicts with standard game pads and user capabilities. we can then use of method as a filter insted. Given a large database we filter stuff that is too different.


We would like to extend our concept of control genomes. Can they be formalized even more ? Could we develop a set of heuristics to extract from animations in a more structured manner ? 
We would like to investigate more primitive types. Further compare how different primitives can be used for various types of locomotion.

More diverse animation tests.


Issue: Depending on steping phase a turn might be faster or slower. For games we want the same reaction time.

\section{Conclusion}
We have presented a method for designing character controllers in computer games that react to user input and generate movement similar to reference animations. Since user control is not inherently restricted we use control genomes to ensure that complexity is either contained in movement models or entirely removed by animation alignment. We suggested a movement model for plane locomotion and applied our method to a set of dance card animations similar to those found in industry productions to demonstrate that auto differentiation can be used both for alignment and parameter estimation.  



