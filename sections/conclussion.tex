\section{Discussion and Limitations}
%In this section we discuss some of the decisions behind our work including why we chose to use simple models composed of primitives, why we chose to do animation alignment and how our method generalizes to other types of animation data. 

%\subsection{Model structure}
Dense models such as fully connected neural networks or radial basis functions have time and again proved capable of learning an impressive amount of details from diverse training data. It is a natural question why we chose to model plane locomotion using a much simpler model? We point to main three reasons. Game developers have extremely tight CPU budgets an in our experience would only allocate microseconds for evaluation of movement models, which is not possible using heavier models. Secondly the black-box nature of especially neural networks makes it challenging to predict the characteristics of the synthesized movement which is important for level designers.  Lastly, we notice that complex models tend to suffer from a lack of consistency where identical inputs produce varying movement patterns depending on the current state of the characters such as step phases. This is natural and even wanted when the aim is only to model realistic animation, but in a game context the user expects a tight and consistent correspondence between input and on screen behavior. 

An analytical model like ours can be of great practical use to the game developers. The parameters have direct links to explainable properties in the animation which makes it easier to tweak the behavior of the model. On the other hand, the analytical approach requires that the designers are able to adequately model the complexity of the movement. 
%
When this is not the case, we imagine applying machine learning techniques to discover latent properties of our animation set. % 
One approach is to extract a limited latent variable set using a dense neural network and then feed the latent variables to a set a primitives which are weighted to give the final output. 
Alternatively, we could apply a hybrid approach by discovering correlations in the data by automated statistical analysis, but still allow the designer to setup the primitives and chose which parameters should be used. 

%\subsection{Generalization}
Our simple model for plane locomotion works well for animations that resemble dance cards, where movements can be easily segmented into straight and turning parts. Additionally motion capture artists and animators have experience in keeping movement consistent in such recordings, which lessens the complexity of the modeling task. We have shown the advantage of aligning even simple dance cards animations, and one might ask how our method generalizes to reference animations where this is not possible due to the visual artifacts that would be introduced? We might image animations with no straight segments at all that continuously shift direction or constant changes in movement speed and turns. While such characteristic are rare for interactive character control, this would pose a challenge. We suspect control genomes would be manually annotated to the animations which might not be practical, and the complexity of the movement models increased. In such cases it might prove more efficient to bypass the movement model entirely, but as we have described this might not be possible due to the practicalities of game productions. In the end we perceive a general limitation in our method that required homogeneous reference animations, but note that this is most likely the case for game development in general. 

%\subsection{Future work}
While we have shown our models can fit well to individual animation segments, it remains challenging to extract global models as the reference animations grow in size. There is an unavoidable challenge to model movement variations, but even the size can pose a challenge for our optimization that can become to slow for practical use. We would like to examine how alternative methods for gradient descent can accelerate the optimization. 
Additionally our simple movement model for plane locomotion is not able to express the variation typically seen in animation when moving beyond tightly controlled dance cards. We would like to examine in more detail, how primitives are best combined and if more types should be added.

\section{Conclusion}
We have presented a method for designing character controllers in computer games that react to user input and generate movement similar to reference animations. Since user control is not inherently restricted we use control genomes to ensure that complexity is contained in movement models or entirely removed by animation alignment. We suggested a movement model for plane locomotion and applied our method to a set of dance card animations similar to those found in industry productions to demonstrate that auto differentiation can be used both for alignment and parameter estimation.  
