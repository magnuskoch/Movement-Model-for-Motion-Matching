\documentclass[format=acmtog]{acmart}

\usepackage{blindtext}

\usepackage{graphicx}
\graphicspath{ {./img/} }

\usepackage{subfig}


\usepackage[ruled]{algorithm2e} % For algorithms
\renewcommand{\algorithmcfname}{ALGORITHM}
\SetAlFnt{\small}
\SetAlCapFnt{\small}
\SetAlCapNameFnt{\small}
\SetAlCapHSkip{0pt}
\IncMargin{-\parindent}


%\newcommand{\abs}[1]{\left| {#1} \right|}
%\newcommand{\acos }{ \ensuremath{ \cos^{-1} } }
%\newcommand{\asin }{ \ensuremath{ \sin^{-1} } }
%\newcommand{\atan }{ \ensuremath{ \phantom{\cdot}\text{atan}_2 } }
%\newcommand{\ball}{\ensuremath{ \mathcal{B} }}
%\newcommand{\bigO}{ \ensuremath{ \mathcal{O} }}
%\newcommand{\degree}{\,^{\circ}}
%\renewcommand{\det}[1]{ \ensuremath{ \text{det}\left( #1 \right) } }
%\newcommand{\hull }[1]{ \ensuremath{ \text{convex hull}\left( #1 \right) } }
%\newcommand{\identity}{ \ensuremath{ \mat I }}
%\newcommand{\idx }[1]{ \ensuremath{ \mathbb{#1} } }
%\newcommand{\intersection}{\ensuremath{ \cap }}
%\newcommand{\logand }{ \ensuremath{ \wedge } }
%\newcommand{\logor }{ \ensuremath{ \vee } }
\newcommand{\mat}[1]{\ensuremath{\mathbf{#1} }}
%\newcommand{\interior}[1]{\ensuremath{ {#1}^{\circ} }}
%\newcommand{\boundary}[1]{\ensuremath{ \partial{#1} }}
%\newcommand{\N}{ \mathbb{N} }
\newcommand{\norm}[1]{\parallel {#1} \parallel}
%\newcommand{\proj }[1]{ \ensuremath{ \text{proj}\left( #1 \right) } }
\newcommand{\prox }[2]{ \ensuremath{ \text{prox}_{#1}\left( #2 \right) } }
\renewcommand{\Re}{ \mathbb{R} }
\newcommand{\Z}{ \mathbb{Z} }
\newcommand{\set}[1]{\mathcal{#1}}
\newcommand{\seq}[1]{ \left\{ #1 \right\} }
%\newcommand{\sgn }[1]{ \ensuremath{ \text{sgn}\left( #1 \right) } }
%\newcommand{\smallO}{\ensuremath{ \mathbf{o} }}
\renewcommand{\th}{ \ensuremath{ ^{\text{th}}} }
%\newcommand{\trace }[1]{ \ensuremath{ \text{tr}\left( #1 \right) } }
%\newcommand{\union}{\ensuremath{ \cup }}
\renewcommand{\vec}[1]{ \boldsymbol{#1 } }
\renewcommand{\quat}[1]{ \boldsymbol{#1 } }
%\newcommand{\activeset}{ \mathcal{ A } }
%\newcommand{\allset}{ \mathcal{ E } \cup \mathcal{I} }
%\newcommand{\arctantwo}{\ensuremath{\arctan\!2}}
%\newcommand{\crossmat}[1]{\ensuremath{\boldsymbol{#1}^{\times} }}
%\newcommand{\curl}{\operatorname{\text{curl}}}
%\newcommand{\diag}{\operatorname{\text{diag}}}
%\newcommand{\divergence}{\operatorname{\text{div}}}
%\newcommand{\enorm}[1]{\ensuremath{\left\| #1 \right\|_{_2}}}
%\newcommand{\equalityset}{ \mathcal{ E } }
%\newcommand{\feasibledirections}{ \ensuremath{\mathcal{F}} }
%\newcommand{\feasibleregion}{ \ensuremath{\Omega} }
%\newcommand{\func}[1]{{\bf{#1}}}
%\newcommand{\grad}{\operatorname{\nabla}}
%\newcommand{\inequalityset}{ \mathcal{ I } }
%\newcommand{\jacobian}[1]{\ensuremath{\boldsymbol{\mathit{#1}} }}
%\newcommand{\lagrangian}{ \mathcal{ L } }
%\newcommand{\rtm}{$^{\textrm{®}}$} % Registred trademark  \textregistered
%\newcommand{\tangentcone}{ \ensuremath{ \mathcal{ T }_{\Omega} } }
%\newcommand{\vecop}{\operatorname{\text{vec}}}
%\newcommand{\FG}{ \mathcal{G} }
%\newcommand{\FC}{ \mathcal{F} }
%\newcommand{\LS}{ \mathcal{L} }

\newcommand{\model}{ \mathcal{M} }
\newcommand{\modes}{ \mathcal{L} }
\newcommand{\control}{ \mathcal{C} }
\newcommand{\interpolators}{ \mathcal{T} }
\newcommand{\actor}{ \mathcal{A} }

\newcommand{\mode}{ l }
\newcommand{\interpolator}{ I }

\newcommand{\pos}{ \vec{p} }
\newcommand{\ori}{ \quat{\theta} }
\newcommand{\dtpos}{ \dot{\pos} }
\newcommand{\dtori}{ \dot{\ori} }
\newcommand{\blend}{ \vec \omega }
\newcommand{\weight}{ \omega }
\newcommand{\sfunc}{ \textit{Step} }
\newcommand{\interp}{ u }

\usepackage{color} 
\definecolor{color1}{rgb}{0.5,0.1,0.1}
\definecolor{color2}{rgb}{0.1,0.5,0.1}
\definecolor{color3}{rgb}{0.1,0.1,0.5}
\definecolor{color4}{rgb}{0.5,0.1,0.5}
\newcommand{\kenny}[1]{{ \color{color1} K: #1}}
\newcommand{\magnus}[1]{{ \color{color2} M: #1}}


\begin{document}
\title{Movement Model for Motion Matching}


%%
%% The "author" command and its associated commands are used to define
%% the authors and their affiliations.
%% Of note is the shared affiliation of the first two authors, and the
%% "authornote" and "authornotemark" commands
%% used to denote shared contribution to the research.

\author{Magnus Koch}
\email{m.koch@di.ku.dk}
\orcid{1234-5678-9012}

\author{xxx yyyy}
\email{zzzz@wwww.dk}
\orcid{1234-5678-9012}

\author{Kenny Erleben}
\email{kenny@di.ku.dk}
\orcid{0000-0001-6808-4747}

\affiliation{%
  \institution{University of Copenhagen}
}

%%
%% By default, the full list of authors will be used in the page
%% headers. Often, this list is too long, and will overlap
%% other information printed in the page headers. This command allows
%% the author to define a more concise list
%% of authors' names for this purpose.
\renewcommand{\shortauthors}{Koch et al.}


\begin{abstract}
%In this paper w
We present a method to extract models from reference animations that can reproduce movement from user input. 
%Although 
Much work has been to done to create realistic full body character animation there is a lack of methods for performing the supplementary task of accurate modeling between user input and future trajectory. This is important for systems that use %\kenny{does reviewer know what this is?} 
weak control signals to guide animation synthesis
%, but e
Especially for computer games where the character position is typically mapped directly to the estimated trajectory with animations overlaid as a secondary effect that is not allowed to diverge. \changed{When a visual discrepancy artifacts appear current practice requires a tedious manual alignment process to avoid this. Our method provides an automated alternative to current practice.} \changed{Input control is ill-defined as the user may not be able to faithfully reconstruct the full movement details using a gamepad. Hence, we regularize the task by introducing control genomes to represent reduced user control signals, thereby deferring complexity to our models.} We further introduce movement models using a modular approach where primitives are combined and we demonstrate a model for plane locomotion in games. Finally the modeling task is further addressed by animation alignment and trajectory estimation where unwanted details are filtered.
\end{abstract}

%%
%% The code below is generated by the tool at http://dl.acm.org/ccs.cfm.
%% Please copy and paste the code instead of the example below.
%%
\begin{CCSXML}
<ccs2012>
   <concept>
       <concept_id>10010147.10010371.10010352.10010238</concept_id>
       <concept_desc>Computing methodologies~Motion capture</concept_desc>
       <concept_significance>500</concept_significance>
       </concept>
   <concept>
       <concept_id>10010147.10010371.10010352.10010380</concept_id>
       <concept_desc>Computing methodologies~Motion processing</concept_desc>
       <concept_significance>500</concept_significance>
       </concept>
   <concept>
       <concept_id>10010147.10010371.10010352.10010378</concept_id>
       <concept_desc>Computing methodologies~Procedural animation</concept_desc>
       <concept_significance>300</concept_significance>
       </concept>
 </ccs2012>
\end{CCSXML}

\ccsdesc[500]{Computing methodologies~Motion capture}
\ccsdesc[500]{Computing methodologies~Motion processing}
\ccsdesc[300]{Computing methodologies~Procedural animation}

%%
%% Keywords. The author(s) should pick words that accurately describe
%% the work being presented. Separate the keywords with commas.
\keywords{Computer Games, Animation Control, Motion Matching}

%% A "teaser" image appears between the author and affiliation
%% information and the body of the document, and typically spans the
%% page.
\begin{teaserfigure}
    \subfloat[]{
    \label{fig:teaser:a}
    \includegraphics[width=1\textwidth]{img/teaser}
    }
    \protect \caption{See my cool graphics illustration of why this is awesome.} 
  \Description{Awesome.}
  \label{fig:teaser}
\end{teaserfigure}



\maketitle

\section{meta}
\subsection{Method section}
\begin{itemize}
    \item Describe animations, annotated control -> genomes.
    \item Describe model. A what can influence movement ? Velocity. Angular Velocity. Sharpness of desired turn. Style (run etc). Movement templates that react to this, and return updated parameters. Position, and orientation. Rest can be derived.
\end{itemize}

\subsection{Elevator Pitch A}
\clearpage
It is often necessary to understand the coverage (range of motions) available in an animation database. However, we do not know of any good metric to answer this question, what is coverage ? We propose to describe coverage through a Movement Model that responds to control input. Thus coverage can be examined analytically by inspection of the model, and coverage is formalized at the capability of an animation database to respond to some user controls.

\subsection{Contributions}
\begin{itemize}
    \item Idea to describe coverage through a model that reacts to control genomes.
    \item Description of model for plane locomotion.
    \item Procedure to fit model under non destructive alterations of animations.
\end{itemize}
\subsection{Applications}
\begin{itemize}
    \item Analytical tool for animators.
    \item Displacement of character in computer games. In-place animation played on top.
    \item Control input to data driven systems such as motion matching or generative neural networks.
\end{itemize}

\subsection{Elevator Pitch B}
In game development it is a common construct to have character position driven by a simple algorithmic model that responds to player input.
\\ 
In-place animations are then played at the position given by the model at any given time.
\\
If the model movement matches poorly with the movement in the animations, the character will appear to float.
\\
We propose a general \textit{Movement Model} and optimization framework that ensures synchronization between animations and model. 

\subsection{Verification}
Remove root motion from animation. Replay in-place animation on top of movement model. If there is no sliding, we have a perfect match.


\section{Introduction}
Realistic and responsive character animation is important for establishing immersion in computer games. The complexity of human movement makes it difficult to construct such an experience by hand. Instead we record reference data using motion capture and construct animation synthesizers that either stitches clips together, samples inferred statistical models, or learns control policies for a physical model to mimic the animations. Recent advances in both industry and research communities show a strong tendency towards complex closed systems guided by weak control signals. (use a spring to), (use a to).
To our best knowledge it has not been investigated how to optimally construct and synchronize weak control signals to full body animation synthesizers. This is important for computer games. It is an industry standard to have a Gameplay Layer that controls changes to character position and a separate Animation Layer that tries to generate animations that matches the changes in positions. We suggest the term \textit{Movement Model} to denote the Gameplay Layer logic that controls character position in response to player input. This model is carefully tuned by designers to give movement response that \textit{feels appealing}, it is use to predict future character movement and even for analyzing the validity of game levels. As such it seems an ideal control signal for animation synthesizers. We have a history of movement available and by design the models computationally cheap to integrate for predictions. But a challenge is introduced. The Movement Model is not only weak signal since but rather the exact position changes of game characters. If there is a disconnect between the Movement Model and the animation synthesizer output we might allow the animations to diverge from the Movement Model and later catch up, but this severely impacts the types of game experiences that can be created and changes the feeling of movement in response to player input. If we enforce synchronization between the two layers artifacts such as foot sliding and a general degradation of realism and quality kick in. It is an industry standard to carefully synchronize animations and movement models by hand at great economical cost.

it is a far less ambitious task to generate Movement models compared to full body animations. But we also have requiremetns to these models. Fast to evaluate (used for prediction etc. 0.0001 ms) . The define the feeling of the game. Should be possible to tweak. 

We introduce 'control genomes' as a wau to regularize the task at hand. Then we suggest some primitives that can be used to build models. And finally we use aniamtion editing techniques in the optimizaation to align the data set . We also discovered a novel way to determine movement in animations based of contact detection that can be of use in its own right.



and   the animation and gameplay code.

Most games calculate changes to character position 

in a gameplay layer which is kept completely isolated from the animation system. 


and the animation layer which produces full body animation to match the movement  





Most game implementations keep strict division between animation and gameplay systems, where the latter infers changes to character position from user input using what we will refer to as the \textit{Movement Model}. By concateniting a history and future predection , we generate trajectoris that can be used as weak control signal for the black box animation synthesizers. But we know have a problem. Signal is no longer weak. It is the exact movement of the charcter. Discrepancy will decrease quality. 

Most games implement a self contained system to control character position independently 
w a system to change character position based on user input. 

In many game productions we have a logical to control the character position from user input. This signal is used as weak control for animation synthesizer. This is ok if we allow the system to drift away from our weak signal. But it becomes a problem if we want the system to precisly follow our guide. CAn be seen in many games. If responsiveness is important, we see lots of sliding (disconnect between movement model and animation synthsizer.)


Disconnect (animatjon and our knowledge of how the system will move)






Both research and industry communities p
Character animation for computer games has embraced a data driven approach in both industry [ref] and research communities [ref]. Animation synthesizers or control policies  



Computer games are growing. Both in scope and ambition. Virtual worlds are becoming larger and contain increasingly complex and dynamic interactions between the characters and the environment. Multiplayer game play is most commonly the norm, and requires that the internal state of the rapidly changing game world can be reliable passed around and kept synchronized between players around the world. To manage the complexity of interaction of the full game world, individual game components are usually treated as abstract models, that should be easy to reason about, to update and keep synchronised. Then as a next step more fine grained models or simply visual fidelity is added on top. 

While it has proven extremely difficult to synthesize dynamic animations of human characters, the challenge is only greater in the context of modern game production. A common approach is to split the animation system into a abstract movement model, which produces the overall state of the game characters, and an animation system which generates animations that realistically and closely follow the path of the movement model. Introduced in 2015 Motion Matching has been adopted by many studios \kenny{as evidenced by games such ass XX, YY, ZZ}. At its core, the system stitches animations streams together by continually transitioning between frames in a large database of animation clips. The transitions are determined by finding the best matches to an external requirement which is naturally provided by the movement model. Motion matching excels when the animation database has a substantial coverage allowing the system to generate a full range of animations that are still close to the ground truth \kenny{Can we add some cites for this claim? Or we can rephrase it as: "In our experience Motion Matching excels...."}. 

Conceptually the movement model provides a control signal into the motion database. To keep synthesized animation realistic and fluent, it is critical that this control signal is reproducible in the database. In practice this is often not the case, and the issue is addressed by manually tweaking all exposed parameters in the pipeline. Most importantly weighted heuristics are added to the motion matching system, animation trajectories are manually edited and the movement model is set to reproduce the main movement modes present in the animations. To our best knowledge the process of tweaking poses is the biggest challenge to the practical use of motion matching.

In this paper we formulate the synchronization of the motion matching system, an animation database and movement model as an ill posed optimization procedure. We suggest a number of regularization steps, arriving at a solution that enables us to use unstructured animations for motion matching under movement model control while preserving fidelity in the database. The regularization procedure has two steps, where we first identify main modes in the animation database, and then distribute a sparse user control along all animation in the database. Manual edits to the animations is replaced by a warping system, allowing us to optimize against a wider range of valid input animations. Finally an optimization procedure is applied to the partially constrained movement model and animations to arrive at a configuration of the entire system.

Our contribution is a conceptual formulation of the movement model, the regularization steps required to run the optimization procedure and the [something] required to make optimization work nicely.

\section{Related work}
It is notoriously challenging to construct general models for human movement due to the complex interplay of planning and optimization processes at work, and it is often necessary to develop ad hoc solutions. Our work relates most closely to path following as in MOTIONMATCHINGREF, \citep{holden.16} or \citep{startke20} where movement is specified explicitly by the user and tracked by the animation synthesizer which is constructed from a set of reference animations. Such systems typically allow a discrepancy between the control signal and the path traced out by the generated animations or avoid the issue by adding more reference data for better coverage. Our method address this discrepancy as a pre-pass where a suitable movement model is found and reference animations are aligned, before being passed on to a secondary system.  

A further motivation for our work is seen in the division between gameplay and animation layers described by \citep{holden18} and reflected in state of the art industry tools such as Unreal Engine and Unity. Both expose animation systems with the ability to toggle root motion for animation clips. In our experience root motion is primarily used for static content such as cinematics, while dynamic gameplay relies on custom logic to provide changes in character position. 

We also relate our method to other research areas where high level controls are translated or used directly to drive an animation synthesizer. 

To our knowledge no work has been published with synchronization between model and reference animations by adjustment of both components.
\subsection{User input attractor}
In the game industry character movement controllers typically transition from a starting state towards the direction supplied by the user using an assortment of smoothing techniques such as lerp, exponential decay or damped springs applied to positions, velocities or accelerations as seen in \citep{buttner20} and \citep{holden21}. 
In research a similar approach is in \citep{mccann07}, \citep{holden.16}, \citep{Zhang18}. Additionally \citep{startke20} projects smooth user input to a lower level control space similar to trajectories found in reference animations. This illustrates the discrepancy between weak control signals and synthesized motion which is common to these methods, and addressed by exposing a parameter to control divergence of synthesized motion from the control signal (responsiveness) versus accepting visual artifacts (realism) MOTION MTACHJING SLIDER REF. 

It is a complex task to design expressive controllers using smooth input attraction. As a generalization research prefers simple models with complexity relayed to the animation synthesizer, while industry need a stronger correspondence and construct elaborate models using multiple springs intricately connected by state machines. Our method proposes to replace this tedious and difficult work through automation. 

A variant of user input attraction is found in \citep{treuille07} where an offline system tracks a path drawn by the user. In this case we have no movement model as the user input is exactly the low level control signal. Oppositely \citep{kovar02} use reward functions to provide less constrained control when the user specifies goals such has 'end position' that have no path requirements, and in \citep{lee18} 'straightness' and other stylistic traits contribute to similar rewards. In these cases the movement model is embedded in the animation synthesizes, but could be separated by augmenting the reward functions. Since high level tasks are used to allow the synthesizer some freedom, the specification is most often kept simple. A hybrid approach in \citep{lee10} track a control direction from the user with no explicit path but with a short horizon tending toward a path specification.

\subsection{Motion planning and control theory}
When a closed dynamical system exposes control input it is necessary to analyze how specific movements can be achieved using the available commands. Motion planning explores manipulation of mechanical systems like robots, while control theory examine systems of differential equations. In both cases the task is to extract discrete control signals that would makes a closed system follow a reference motion. This is inverse to the analysis of movement models, where we seek to replicate a closed system by examining relationships between fixed control signals (control genomes) and movement extracted from reference animations. 

\subsection{Function approximation}
Similar to \citep{startke20} it is possible to augment a basic input attractor with a secondary filter to capture additional details in reference animations. Dynamical movement primitives activate radial basis functions with a secondary scale function to capture such details while remaining robust to changes in timing. This method could replace our movement model, but requires an increasing amount of basis functions to capture finer details. This makes the model difficult to inspect and expensive to evaluate. Furthermore our approach manages complexity in a dual approach by combining movement models that are more expressive than simple point attractors, with alignments that remove unnecessary complexity from the reference animations. This retains the correctness of the end-to-end solution, but eliminates the need to capture fine grained details in the movement model.




\section{System Overview}
Human movement is a complex interplay between intentions within the brain and the physical environment. The though to 'walk-forward' triggers and array of intractable optimization  

In the following we will introduce a method for optimal synchronization between a Movement Model and a set of animations given Control Genomes. There are 4 main parts to the system. 
\begin{itemize}
    \item \textbf{Control Genomes} extracted from animations.
    \item \textbf{Movement Model} that generates a movement given Control Genomes.
    \item \textbf{Non destructive} animation smoothing. 
    \item \textbf{Optimization procedure} that fits exposed parameters of all the above systems to minimize the difference between the movement- and animation trajectories.
\end{itemize}
Figure \ref{fig:movement:model} illustrates the definitions and concepts of a movement model.
\begin{figure}
    \centering
    \includegraphics[width=0.75\columnwidth]{img/method-overview.png}
    \caption{Theta values indicate parameters subject to optimization. Goals are fixed to add regularization}.
    label{fig:movement:model}
\end{figure}

\section{Control Genomes}
We define Control Genomes as signals that can generate movement. They are a combination of intentions and enough contextual state to follow the Markov principle. As an example a position, direction and time is a Control Genome for a straight walk. Conversely a straight walk contains a latent Control Genome (position, direction, time). Under this naive framework we quickly realize that multiple straight walks could be associated with a single Control Genome. We further impose the constraint that Control Genomes must disambiguate movement, ie. a unique Control Genome maps to a unique movement. In our simple example we might resolve conflicts by adding a style parameter and assign values such as 'brisk walk' or 'dragging feet' to our Control Genomes. 

Control Genomes can have a direct counter part in the host application such as user input through a game controller or navigational path from an AI system, and therefore can also be viewed as tasks (ie. turn right) that are carried out by the animations. In general we want the Control Genomes to be of minimal size, since in the limit we could have the animation itself as the Control Genome. We say that a genome is in \textit{reduced} and \textit{segmented} form if it contains no repetitions and removal of information would break either the Markov property of the disambiguation constraint. 

\begin{figure}
    \centering
    \includegraphics[width=1\columnwidth]{img/controlgenome.png}
    \caption{Control Genome}
    \label{fig:control:genome}
\end{figure}

Figure Fig. \ref{fig:control:genome} shows an example where an animation has been assigned a Control Genome. Each frame is annotated with a 2d-control direction corresponding to stick input from a game controller. A dense and redundant Control Genome contains the entire skeletal animation as well as the annotated directions. The animation has two similar right turns so we get a single segmented Control Genome by splitting the redundant genome in two identical parts. We then trim all unnecessary information from the genome to get the reduced form which only contain a starting position and two direction offset in time. This last step assumes that we are capable of regenerating the animation from the reduced genome using a procedure we will refer to as a \textit{Movement Model}. A [Control Genome, Movement Model]-pair describes an animation exactly when the Movement Model can regenerate the Animation from the Control Genome.  In practice it is not possible to develop accurate generative or predictive models for human movement. So we will allow our animations to undergo non destructive transformation such as smoothing and path adjustment, and only expect our model to regenerate a fraction of the complete signal, such as the trajectory. 

We will now proceed to a formalized description of Control Genomes.

The input to our system is an annotated animation. Let $\anim^{\dimas,\dimaa}_\dimat$ denote $\dimat$ frames of animation of a skeleton with $\dimas$ degrees of freedom, where each frame is given an $\dimaa$-dimensional annotation. We assume the presence of a Retrieve-and-Collapse procedure (automated or manual) that extracts all reduced and segmented Control Genomes present in the animation and for each maintains a pairing to all $k$ corresponding (un-annotated) segments of the source animation. With $i$ as the index and $\dimg$ as the dimensionality of the Control Genomes we have $\genome^g_i\in\reco(\anim^{\dimas,\dimaa}_\dimat)\rightarrow\anim^{\dimas,0}_{\dimae_0}\ldots\anim^{\dimas,0}_{\dimae_k}$ where $\dimae\ll\dimat$ refers to a segment of the full animation and $k$ is the number of segments matched to the $i$'th Control Genome.


We define two parametric projections to a shared $\dimes$-dimensional evaluation space as $\model(\paramm,\genome^{\dimg}) \rightarrow \mathcal{R}^{\dimes}$ and $\edit(\parame,\anim_{\dimae}^{\dimas,0}) \rightarrow \mathcal{R}^{\dimes}$. The evaluation space will usually have a natural counterpart in the application such as the trajectory (position and orientation of character over time).\\
Intuitively $\model$ is a Movement Model capable of generating movement (or more accurately an evaluation space representation) from the Control Genomes. $\edit$ corresponds to the adjustment we allow our animations to undergo, such a smoothing or more advanced manipulation. The optimal parameters for a single Control Genome are found as the minimization of the L2-norm.
\begin{subequations}
\begin{align}
    \gnorm(\paramm,\parame,\genome)&=\sum_{\anim_k\in\genome}{\frac{1}{2}|\model(\paramm,\genome)-\edit(\parame\anim_k)|^2}\\
    \paramm^*,\parame^*&=\min_{\paramm,\parame}{\gnorm(\paramm,\parame,\genome)}\label{eq:optim:single}
\end{align}
\end{subequations}
When we want constant parameters across the fitting of multiple Control Genomes $\genome_i\in\reco(\genome)$ an additional term is added to the minimization. 
\begin{equation}
    \vec{\paramm^*},\vec{\parame^*}=\min_{\paramm,\parame}{\sum_{\genome_i\in\reco(\genome)}{\gnorm(\paramm^i,\parame^i,\genome_i)}}+VAR(\vec{\paramm},\vec{\parame})
\end{equation}
Where $\vec{\paramm^*},\vec{\parame^*}$ is the full set of parameters and $\paramm^i,\parame^i$ are the parameters fitted to $\genome_i$.

\section{Movement Model Take 2}
In this section we will describe how movement can be generated from Control Genomes using a Movement Model. We will examine plane locomotion and limit the output of the model to trajectories ie. time series of connected positions and facing directions. 
Conceptually there are no limitations on the complexity of the models we chose. Dense neural networks [ref] or muscle based physical simulations [ref] are capable of even generating full body locomotion. However we would like a model that exposes easily and exactly tweakable parameters to the game designers and animators. In many game scenarios micro timings and the 'feel' of the character movement are core parts of the user experience. Accordingly we need descriptive models, that can still be transparently manipulated by the artists. 

As described in the previous section a Control Genome consists of an initial state paired with a sequence of control inputs. Let $\pos_{t=0},\facing_{t=0}$ describe the initial position and facing angle of the character. Player input is supplied through standard gamepad control using stick direction and button presses with $\controlface_t, \controlmove_t$ representing the desired facing and movement directions at time $t$. Further parameters can be added to support different movement styles.   

A sequence of movement is generated by recursive updates to the genome state by the movement model.

\begin{subequations}
\begin{align}
    \pos_{t+\dt},\facing_{t+\dt}&\leftarrow[\pos,\facing,\controlface,\controlmove]_t,\dt\label{eq:move:update}
\end{align}
\end{subequations}
The update  acts as motion planning by controlling how the current state gradually transitions towards a new state as defined by the Control Genome. The objective function in \ref{eq:optim:single} imposes a restraint to model this transition process according to the characteristics of the reference animation. 

We construct the planning function in \ref{eq:move:update} as a composition of planning primitives. Each primitive exposes a set of adjustable parameters and performs input to output mapping using various interpolation methods. In the limit the compositions could approximate full neural networks, but should be kept simple enough for human manipulation of each parameter while maintaining the capability to model the movement in the reference animations.     

Fig [??] illustrates the 3 planning primitives we use to model plane locomotion. A Critically damped spring $out\leftarrow{}\spring{(in,t,k)}$ exposes a spring coefficient $k$ used to control the drag of the input variable towards the target $t$. The 1D-Map $out\leftarrow{}\mapo(in,k)$ uses spline interpolation over knots in $k$ to map a parametric input to a line position. The 2D-map $out\leftarrow{}\mapt([in_x, in_y], k)$ uses barycentric coordinates to interpolate polygon knots in $k$ according to a 2 dimensional grid position given as input.    

As an example we propose a Movement Model for plane locomotion typical for 3rd person computer games that strikes a compromise between simplicity and expressiveness. We use the convention that s, c, m, i refers to values associated with state, control, model output, and  model intermediaries respectively. First the position and facing information contained in the Control Genome state is augmented with derived values $\statespeed, \statemove, \stateangularspeed$ for speed, move direction and angular speed ie. rate of change of the movement direction. We also compute $\diffmove=||\statemove-\controlmove||$ as the absolute error between the state and control values for movement direction. A Movement model is constructed by first describing changes related to speed.

\begin{subequations}
\begin{align}
    \modelspeed&\leftarrow{}\spring(\statespeed,\intermediaryspeed,k_\intermediary)\\
    \intermediaryspeed&\leftarrow{}\mapt([\stateangularspeed,\diffmove],\param_0)\\
    k_\intermediary&\leftarrow{}\mapo(\statespeed-\intermediaryspeed, \param_1)
\end{align}
\end{subequations}
Here $\param_1$ models the acceleration of the model depending on the relative change in speed, and $\param_0$ models planning of the target speed depending on the current angular speed and the required change to movement direction.

Modeling of changes to movement and facing directions are added.
\begin{subequations}

\begin{align}
    \modelmove&\leftarrow\spring(\statemove,\controlmove,\mapt([\statespeed,\diffmove], \param_2))\\ 
    \modelfacing&\leftarrow\spring(\statefacing,\controlface,\mapt([\statespeed,\difffacing], \param_3))
\end{align}
\end{subequations}
Here $\param_2$ and $\param_3$ model planning of movement and facing direction dependent on the current movement and facing values and the differences to the control targets.


If we use 9 knots for the 2D blends, and 4 four the 1d, we have a total of 4 + 3*9 tuneable parameters. An example plot of the model using reasonable parameters is depicted in fig ??.


Introducde, diff values. Control fiff.


Variation comes from implementing multiple plannining functions and transitioning between those using something like a gating network.

\section{Adjustment Take 2}
\section{Optimization Take 2}

\section{Movement Model}
A movement model describes how an entity moves through space given control input (goals). It consists of internal parameters such as velocities and accelerations. Control input could be a direction of movement supplied by a human using a gamepad, or longer paths supplied by a game engine AI system. We have not seen any formal description of the movement model. Usually some ad hoc newtonian physics concepts are combined with springs and animation curves to control the system \kenny{Thing about generality of your statement, as it stands now you are claiming this is so for all. Hence, can you prove/cite it? Otherwise rephrase so it becomes a more subjective observation from a single context. Like to our knowlege or our observation suggest that.}.

For complex systems such as human beings the movement model has an inherent dichotomy between accuracy and simplicity. Deep Neural Networks are hard to tweak after they have been trained, physical simulations are hard to tune towards artistic expressions while on the other hand a 2-parameter linear model will be too simplistic to capture meaningful details.



In the following we limit \kenny{without loss of generalization} the description to locomotion in the plane. 
\subsection{general frame work}
We parameterize the movement model as 
\begin{equation}
    \label{eq:model:parameterization}
 \model \equiv 
 \left\{ \modes,\,\interpolators \right\}
\end{equation}
where $\modes \equiv \seq{ \mode_1, \ldots, \mode_n}$ is a set of locomotion modes and $\interpolators \equiv \seq{ \interpolator_1, \ldots, \interpolator_m}$ is a set of locomotion mode interpolators that describes the transition between locomotion modes. We use the convention that $\interpolator_j^{a\rightarrow b}$ where $j$ refers to linear index into $\interpolators$ and $a,b$ refers to indices in $\modes$ respectively, ie. $\interpolator^{a\rightarrow b}$ refers to the interpolation between $\mode_a$ and $\mode_b$. A movement model defines the behavior of an actor. 
\begin{subequations}
\begin{align}
    \label{eq:actor:def}
    \actor 
    &\equiv
    \seq{
        \blend, \kine 
        }\\
    \kine
    &\equiv
    \seq{
        \pos, \speed, \move, \dtmove, \facing, \dtfacing 
        }\,,
\end{align}
\end{subequations}
Here $\blend$ are contribution weights for different locomotion modes and $\kine$ encapsulates kinematic properties. $\pos$ and $\speed$ describe position and speed (magnitude of velocity), and movement and facing orientations are quaternions $\move$ and $\facing$ with derivatives $\dtmove$ and $\dtfacing$. Fig. \ref{fig:actor} show a visualization of an $\actor$ in a 3d coordinate system.  
\begin{figure}
    \centering
    \includegraphics[width=0.75\columnwidth]{img/actor.jpg}
    \caption{Actor. Not showing derivatives and facing orientation}
    \label{fig:actor}
\end{figure}
\kenny{To make this more concrete one can for example think of $\pos$ as the planar 2D position of the character root position in the world and $\ori$ could be the world facing direction of the character given as an angle measure in the 2D world plane.} The contribution weights for different locomotion modes are given by the vector $\blend \equiv \begin{bmatrix}\weight_1 & \ldots & \weight_n \end{bmatrix}^T$ where $\weight_l$ contains the contribution for $l \in \modes$. We impose that $\norm{\blend}^2 = 1$ and remark that usually only a couple of modes will be active as in a transition between walking and running where 
\begin{equation}
    \label{eq:contribusion:weight:example}
    \weight_l \equiv
    \begin{cases}
    0.8 & \text{if $l$ is walking} \\
    0.2 & \text{if $l$ is running} \\
    0 & \text{otherwise}
    \end{cases}\,.
\end{equation}
To describe changes in $\blend$ over time increments $\dt$ each $\interpolator \in \interpolators$ defines a mapping.\kenny{I do not get the notation here, $(\cdot, \cdot)$ is a mapping, maybe it should be written that $\interpolator_{i}^{a,b} : \Re \mapsto \Re^n$ and then one would write $\dweight_{i,b} \leftarrow \interpolator_{i}^{a,b}(\dt)$?} 
\begin{equation}
\dweight_{i,b} \leftarrow (\interpolator_{i}^{a,b},\dt)
\end{equation}
where $\dweight_{i,a}=-\dweight_{i,b}$. \kenny{The sub-index is not really needed? I think you can drop it? The minus sign comes from $\interpolator^{a,b}$ being symmetric opposite to $\interpolator^{b,a}$?}

The locomotion modes and interpolators are activated through high level goals provided by user control input or an AI navigation system. The goals contain no motion planning and should provide only states that we wish to active as fast a possible. Goals are highly context dependent, and for locomotion in the plane we use the following.
\begin{subequations}
\begin{align}
     \goals &\equiv \seq{\mode_g, \kine_g} \label{eq:goals:def} \\
\end{align}
\end{subequations}

where $\mode_g$ and $\kine_g$ is the requested locomotion mode and kinematics respectively. We also use $g$ as subscript for blend weights of the requested locomotion mode.

The sparsity of the goal specification illustrates an inherent uncertainty in character animation, as we are challenged to infer a detailed path of movement through often complex environments given a very limited disambiguation or hints to the desired trajectory \cite{holden.ea16}. We notice that a sampling of the immediate surroundings could potentially be added as part of the goals as in \cite{holden.ea17}. 

A step of the actor state can be described by $\actor_{t+1} = step_{\actor}(\actor, \modes, \interpolators, \goals)$ and split into two separate functions to make the update process clearer.
\begin{subequations}
\begin{align}
    \blend_{t+1} &\leftarrow \sfunc_{\blend}( \blend_{t}, \interpolators, \mode_g, \dt) \label{eq:step:interpolators}\\
    \delspeed, \delmove, \delfacing &\leftarrow  \sfunc_{\kine}(\blend_t, \kine_{t},  \modes, \kine_{g}, \dt )\label{eq:step:modes}
\end{align}
\end{subequations}
Intuitively $\sfunc_{\blend}$ moves $\blend$ towards $w_g=1$ and $w_{i\neq{g}}=0$ using the combined interpolation rules in $\interpolators$. Since multiple interpolators can be active at the same time, we define the contribution $\weighti_k$ of $\interpolator_k$ as the weight of the locomotion mode $k$ relative to the weight of all locomotion modes we are interpolating out of. 
\begin{equation}
\theta_{k}=\dfrac{\weight_k}{\sum{\weight_n}},
\quad \forall k, n \neq g \,.
\end{equation}
The combined effect of multiple active interpolators on $\weight_g$ follows.
\begin{equation}
\dweight_g = \sum \weighti_k \dweight_{i,k} \,,
\quad \forall k \neq g \,.
\end{equation}
We set $\weight_{g,t+1}=\weight_{g,t}+\dweight_g$ and need to maintain both $||\blend||^2=1$ and the distribution of the locomotion modes we are transitioning away from. Both is achieved by.

\begin{equation}
\weight_k = \theta_k \, (1-\weight_g)  
\end{equation}
As en example, the table shows a transition to locomotion mode $\mode_g$ starting with $[\mode_i,\mode_j]$ as active. The transition uses two interpolators $[\interpolator^{i,g},\interpolator^{j,g}]$ in the timespan $0<=t<=2$.
\begin{center}
 \begin{tabular}{||c c c c||} 
 \hline
 t & $\weight_i$ & $\weight_j$ & $\weight_g$  \\ [0.5pt] 
 \hline\hline
 0.0 & 0.4 & 0.6 & 0.0 \\ 
 \hline
 1.0 & 0.2 & 0.3$\bar{3}$ & 0.5 \\
 \hline
 2.0  & 0.0 & 0.0 & 1.0 \\
 \hline
\end{tabular}
\end{center}

\magnus{
\begin{center}
 \begin{tabular}{c c} 
 $(\interpolator^{i,g},1.0)=$ & 0.65\\ 
 $(\interpolator^{j,g},1.0)=$ & 0.4\\ 
\end{tabular}
\end{center}
}

To evaluate $\sfunc_{\kine}$ we notice that each $\mode\in\modes$ should provide update routine $\delspeed_t, \delmove_t \delfacing_t \leftarrow \efunc(\mode, \actor_t,\goals_t)$ which are then weighted.
\begin{subequations}

\begin{align}
\delspeed_t, \delmove_t \delfacing_t &\leftarrow \sum_{k=1}^{n}\weight_k \efunc(\mode_{k},\actor,\goals)\\
\end{align}
\end{subequations}
And used to integrate $\kine$.

\begin{subequations}
\begin{align}
    \pos_{t+\Delta t} &\leftarrow \pos_t + \int_{t}^{t+\Delta t} \dtpos_t\,dt \,, \label{eq:time:int:pos}\\
    \ori_{t+\Delta t} &\leftarrow \ori_t + \int_{t}^{t+\Delta t} \dtori_t\,dt\,, \label{eq:time:int:orientation}\\
    \kine_{t+\dt} &\leftarrow (\kine_t, \pos_{t+\Delta t}, \ori_{t+\Delta t})
\end{align}
\end{subequations}
A simple forward Euler scheme suffices is our choice due to simplicity and performance.

A trajectory is generated by inputting a time series of goals to a Movement model which is then unrolled recursively over time.
\begin{equation}
    [\kine_0 \ldots \kine_t] \leftarrow Unroll(\model,[\goals_{0} \dots \goals{t}])
\end{equation}
For analysis we are not interested in the full $\kine$ state, but only the position and facing orientation in $(\pos, \ori_f)$ which are both available in $\kine$. 
\\\\
To achieve expressive modes we will compose each individual locomotion mode from movement templates. Each template should be easy to understand and modify. Consider the example of running. This would consist on 1 movement template for general locomotion, but then sharp turn might have such a significant strategy that we need more modeling flexibility within the locomotion mode. We would then include a 'sharp turn strategy' as a movement template. Another approach would be to separate running into two distinct locomotion modes (running vs. sharp turn running). However, this would complicate controls, since the user would need to specifically request running or sharp-turn-running. 


\subsection{Trajectory Filter}
An animation is a time series of bone positions and rotations ordered in a hierarchy or skeleton and changing over time.  
\begin{subequations}
\begin{align}    
    \skeleton=&[(\vec{v}_{0,0},\quat{q}_{0,0}), \dots, (\vec{v}_{0,b},\quat{q}_{0,b})]\\
        &\dots \\
            &[(\vec{v}_{t,0},\quat{q}_{t,0}), \dots, (\vec{v}_{t,b},\quat{q}_{t,b})]
\end{align}    
\end{subequations}
A filter transforms an animation to a single 'track'. 
\begin{equation}
    [\vec{v}_0,\quat{q}_0 \dots \vec{v}_t,\quat{q}_t] \leftarrow \ffunc(\skeleton)
\end{equation}
A trajectory is a measure of the overall movement described by the animation. Various heuristics exist to extract trajectories from animations such as projection of hip position to movement plane, plane normal to triangulation between hips and shoulders, estimation of center of mass etc. Each method can be viewed as a filter on the full animation.\\\\
To determine whether such a filter produces a good trajectory, we need a measure of success. This is often not made explicit. We have.
\begin{itemize}
    \item Reproducible by some simpler model
    \item Minimize divergence from full skeleton 
\end{itemize}
The first measure is the one most commonly addressed, as it is common to use a simple spline curve as control input to a generative animation system. Now since the requirement is that the spline matches the trajectory of the source animation, we could for instance apply smoothing to the projected hips to achieve a signal that more resembles a perfect spline, that the raw motion of the hips would.\\
The seconds measure implies that we are not free to apply any filtering since the trajectory is used to position a character. Visual coherence such as positioning and pose blending brakes down if the trajectory is modified without constraint.\\
Mathematically we can formulate the measures by.
\begin{equation}
    \epsilon_a=\min_{\theta_f,\theta_m,{\goals_{0\dots{t}}}}\sum{||\ffunc(\theta_f,\skeleton)-\ufunc(\model(\theta),\goals_{0\dots{t}})||^2 }
\end{equation}
\magnus{How do we formulate second meassue ? If two frame are similar in source animation. They should remain similar when trajectory is used as reference instead of hips. Enforces consistency in way trajectory is estimated relative to similarity expressed by animation it self.}



success criteria. How good does it describe position of character ? How easy is it to replicate ?


\subsection{Optimization}
We chose auto differentiation as our optimization framework. \magnus{motive ?}. 
\\\\
Now, this requires the all our building blocks are differentiable across the parameter space.
\\\\ 
Our problem is ill posed. If we allow any type of $\goals$ (control signal) we could make any type of movement model fit our animations. We want controls to correspond to realistic user input. So we do a pre-pass over our animation  with a heuristic that computes plausible goals at certain key frames. We maintain goals between key frames.
\\

We also do a heuristic pre-pass over the animation to estimate locomotion mode at each goal keyframe.
\\
Finally we apply a heuristic to identify where movement templates should be applied.
\\
We use \magnus{term from reinforcement learning. Start with easy examples, progress to harder}\kenny{curriculum learning?}. We fit all segments in the animation with individual  locomotion modes (including movement templates).
Now we have two further steps.
\begin{itemize}
    \item \textbf{Align} fitted locomotion modes. 
    \item \textbf{Collapse or discard} locomotion modes.
\end{itemize}


Let's go through an example of the optimization. Imagine we have 4 steps, in an animation containing 2 locomotion model. And goals key framed at 1 and 3. For the sake of simplicity our goals only define a direction in the plane instead of full kinematics state $\kine$.
\begin{subequations}
\begin{align}
    \goals_0&={0,[0,1]}\\
    \goals_1&=\goals_0\\
    \goals_2&={1,[1,0]}\\
    \goals_3&=\goals_2
\end{align}
\end{subequations}
So we are requesting locomotion mode 0 at the first two keyframes, and want to move along the y-axis in the plane. The last two requests locomotion mode 1 and moves along the x-axis.
\\
Since our goals are fixed we can precompute $[\weight_1 \ldots \weight_4]$ using $\sfunc_{\omega}$.
\\
Then we unroll the movement model.
\begin{subequations}
\begin{align}
    \kine_1 &=\sfunc_{\kine}(\blend_1, \kine_0, \modes, \kine_{g,1}, \dt )\\
    \kine_2 &=\sfunc_{\kine}(\blend_2, \kine_1, \modes, \kine_{g,2}, \dt )\\
    \kine_3 &=\sfunc_{\kine}(\blend_3, \kine_2, \modes, \kine_{g,3}, \dt )\\
    \kine_4 &=\sfunc_{\kine}(\blend_4, \kine_3, \modes, \kine_{g,4}, \dt )\\
\end{align}
\end{subequations}
\\\\
Use warp and filter to extract 4 trajectory points. Define optimization.

\subsection{Plane Locomotion}
Examine some concrete models. What parameters do they have ? 
\\\\
 Movement templates: Turn strategy and maybe vaulting strategy as an extension.

\subsection{regularization REST OF TEXT IS JUNK}

\magnus{A movement model outputs a control signal. If we add more and more details to this signal, in the end we will output the entire animation. Some system output a path augmented with footsteps. If we augment with more details we move towards full synthesis by movement model. Balance between descriptive and understandable}

All 

Define an animation.
Define a trajectory extraction
Purpose of trajectory is dual . Stay consistent. Similar animation should have similar trajectories. Re replicatable by movement model. 
Define realization of movement model. 
Define optimization.


How do we fit a movement model ? What is the optimization criterion ?

\subsection{Plane locomotion}

This is the general framework. Now we define locomotion modes and transitions. An actual implementation of the movement model

For $\sfunc_{\vec \omega}$ to keep the transitions defined in $\Delta{t}_c$ updates, each $\interpolator\in\interpolators$ should define a mapping $\Delta{t} \rightarrow \Delta \interp$ which also describes the duration of that transition since $\interp=1$ implies a completed interpolation. Each individual $t$ can be modeled uniquely or in a unified approach, using animator supplied curves, Sigmoids, linear interpolation or even a neural network to capture more subtleties in the transitions. In the case of transitions that are interrupted, we simply freeze existing transitions, and perform the incoming transition as a weighted combination of multiple transitions as shown in Figure \ref{fig:frozen-transition}. 
\begin{figure}
    \centering
    \includegraphics[width=0.75\columnwidth]{img/frozen-transitions}
    \caption{Frozen transition \kenny{Describe the take home message that reader should get from the figure}.}
  \label{fig:frozen-transition}
\end{figure}

By freezing and combining transitions in the case of interruptions, we are effectively approximating missing areas of the locomotion mode manifold by interpolations. This could be avoided by expanding $\interpolators$ to also contain transitions between combination of locomotion mode, or by expanding $\modes$ for a wider sampling of the manifold.

 using $\Delta{t}$. As before the update routines can be arbitrarily complex, which is natural given the idiosyncrasy of human movement. As such our choice of expressiveness in these function are imposing limits on the types of locomotion we are able to model. We use a simple yet expressive approach common to game development.   

velocity, velocity damper function, movement spring constant, orientation spring constant

velocity magnitude depends on the amount angular rotation. Slower when curving
move from current velocity to requested velocity is handled by spring.
move from orientation to requested orientation handled by soring.
Parameters are velocity, velocity damper function, velocity spring constant, orientation spring constant.


Notice that the formulation for $D$ is generic. In production a mapping between the generic movement model and a more context specific model would usually be needed. 

\subsection{missing}
Add animator constraint to model ? Example is 180. We dont start moving backwards immediately. First we rotate 90 degrees on the spot and then we start a 90 degree run to idle movement
Handle this by setting limits on rotation relative to forward movement ? Better to have a locomotion mode where velocity magnitude is 0 when facing relative to direction i < threshold.

Clear description of entire parameter set

Show very clear example. Pseudo code with idle and walk state. 



\section{Basic terminology}
Define animation database as $\mathcal{D}$

Define Movement model

Define Animation warping.


\section{Animation Warping}
We need a warping system that preserves ground truth. And some linear metric for divergence than can be scaled up from 0.


\section{Optimization Procedure}
We need some regularization to ill posed problem
\subsection{Primary movement modes}
Idea: Identify areas in the animations with cyclic movement for some duration. Cluster these segments into buckets with some threshold. We now have the velocity and number of main movement modes.
\subsection{Sparse user input}
If we allow extremely high frequency changes in the user input a wide variety of movement model configurations could follow the animation trajectory. We distribute sparse changes in user input over the animations, ie. few keypoints where we identify changes in the user input.



\section{Experiments}
% \begin{figure*}
%     \centering
%     \includegraphics[width=1.0\linewidth]{img/estimated_trajectory_examples.png}
%     \caption{Top down view of 4 animation clips. Hip movement shows oscillations on straight paths and counter steering during turns. Estimated trajectories are cleaner and corresponds well with overall movements in the animations while excluding unnecessary details of to human locomotion characteristics.}
%     \label{fig:results:estimatedtrajectory:examples}
% \end{figure*}

\begin{figure}
    \centering
    \includegraphics[width=1.0\columnwidth]{img/estimated_trajectory_examples_small.png}
    \caption{Top down view of 4 animation clips. Hip movement shows oscillations on straight paths and counter steering during turns. Estimated trajectories are cleaner and corresponds well with overall movements in the animations while excluding unnecessary details of to human locomotion characteristics.}
    \label{fig:results:estimatedtrajectory:examples}
\end{figure}
%\magnus{NOTE: Video: SHow animation using capsule movement with simple exponential decay and original animations . Does not look nice. Then our solution} 
%\kenny{Do you write somewhere that videos are actual in-game captures, so what you see is exactly what a game will deliver in terms of performance and motion quality? Might be good to make a point out of that we have done absolutely no post-processing of the data, it is really the raw-method that gives this quality.}
We recorded a series of animations with the XSens inertial sensor suit on 3 different actors. The raw sensor data was cleaned using post processing available in the XSens MVN Animate Pro software. We recorded movements typically seen in 3rd person computer games by following industry standard 'dance cards', which are lists of movements that in total covers the expected range of motion of the player character. This style of animation differs from unstructured animation by having clearly defined segments such as 'forward-run' or '$45^o$-degree-walk-turn'. In a production setting the dance cards animation will usually be carefully cleaned and aligned by hand, but since our method addresses exactly those steps we analyze the raw animation data. As seen in the video some artifacts are present in the source recordings due to the lack of clean up.

\subsection{Trajectory Estimation}
We estimate trajectories by connecting foot contact center points as described earlier. We detect foot contacts by manual tuning of threshold values for speed and height of the feet. In our experience globally valid thresholds are easily found, since only a single start-contact point is needed without great precision.Overall the method works well, and is robust to the movements seen in our animations. Examples of the estimated trajectories are plotted against ground projected hip movement in Fig. \ref{fig:results:estimatedtrajectory:examples} and the detected foot contacts are marked. While it is challenging to quantify the quality of the estimates we notice several desirable properties such as a lack of oscillations on straight segments and turn curves with tangents similar to the general direction of movement. Fig. \ref{fig:results:estimatedtrajectory:stats} further illustrates that changes in direction (as indicated by spikes in the angular velocity plot) show a clear pattern  which is not evident in the hip movement alone. As such the hip or center of mass movement can be seen as trajectories occluded by a noise component due to human locomotion characteristics. Notice how the hip movement shows repeated and opposite angular velocities for each foot contact, while our trajectories have either no changes or spikes without a pattern, where the latter indicates natural corrections done by the actor to stay on a straight line. 

To avoid sharp trajectory changes on foot contacts we apply a Gaussian filter with a constant standard deviation in a $0.1$ second window around each contact. This filtering is not applied to remove noise from the signal as in traditional trajectory estimation techniques and does not require any tuning as long as the window size is kept smaller than the distance between consecutive foot contacts. The effect is shown in Fig.\ref{fig:results:estimatedtrajectory:stats}.   
%\magnus{Maybe show 4 frames of an animated character with a capsule drawn on top to illustrate visual effect of consistent trajectory}
%\magnus{Maybe show velocity profiles ? Ie we not have smooth movement, but smooth velocities/ constant accelerations}

\begin{figure}
    \centering
    \includegraphics[width=1.0\columnwidth]{img/estimated_trajectory_angular_velocity.png}
    \caption{Angular velocity profiles of 2 animation clips with foot contacts indicated by vertical dashed lines. Top: Straight movement. Center: Same straight movement, but zoomed. Bottom: A turn. Estimated trajectories show clear peaks and smaller integrals of angular velocities compared to projected hip movement. Positive and negative angular velocities indicate right and left steering respectively. The estimated trajectories of straight animations show clear localized non-cyclic adjustments to heading, while hip movement obfuscate details.
    %\\
    The smaller plot to the bottom right shows the effects of applying local smoothing around foot contacts.}
    \label{fig:results:estimatedtrajectory:stats}
\end{figure}


\subsection{Control Genome extraction and movement alignment}
Our animations are based on dance cards with clearly defined regions and repeating straight-turning-straight segments. Turns are in the range from $0^o-180^o$ and we captured a variation of both walking and running from our 3 actors. The clear patterns makes it easy to extract reduced control genomes where an initial position and velocity pair is combined with directions offset in time. Fig. \ref{fig:results:genome_extraction} illustrates how a control genome is extracted from a single left turn using a simple heuristic. If more complex animations are used we imagine that statistics of recorded stick movements during gameplay or manual annotation could be used to construct control genomes. 
\begin{figure}
    \centering
    \includegraphics[width=1.0\columnwidth]{img/genome_extract.png}
    \caption{Control Genome is extracted from an animation where the actor runs forward then does a $180^o$ turn and runs straight again. The animation is segmented into parts of either straight (black) or complex (green) movement as illustrated in the top right image. We replace turning sections, with the first entry from the proceeding straight section. Entries marked with a solid square contain repeated information and are extracted only once with a timestamp.}
    \label{fig:results:genome_extraction}
\end{figure}

After genome extraction we defined straight animation sections to begin when the animation velocity is parallel with the current control genome direction, and end when a new direction is encountered as previously described. Each straight segment is fitted to remove minor divergences from the straight line.
% \begin{figure}
%     \centering
%     \includegraphics[width=1.0\columnwidth]{img/movement_stats.png}
%     \caption{An animation with 5 90 degree turns has complex changes to speed and angular speed.}
%     \label{fig:results:trajectory_estimation}
% \end{figure}
% \begin{figure}
%     \centering
%     \includegraphics[width=1.0\columnwidth]{img/trajectory_estimation.png}
%     \caption{Left plot shows hip movement cross back and forth between the estimated trajectory, and tangential directions sampled at individual positions hip give poor descriptions of the overall movement direction. Right plot show trajectories having more stable velocities.}
%     \label{fig:results:trajectory_estimation}
% \end{figure}

% \begin{figure}
%     \centering
%     \includegraphics[width=1.0\columnwidth]{img/trajectory_estimation_smooth.png}
%     \caption{By interpolation frame delta across foot contact dependent section, we achieve a smoother sampling of the trajectory. The smoothing is visible in both the path and the velocities.}
%     \label{fig:results:trajectory_estimation_smooth}
% \end{figure}

% \begin{figure}
%     \centering
%     \includegraphics[width=1.0\columnwidth]{img/straight_trajectory.png}
%     \caption{}
%     \label{fig:results:trajectory_straight}
% \end{figure}
\subsection{Movement Model Fitting}
We performed local fittings on our animations using control genomes with 2-5 changes in direction. The optimization was performed using PyTorch and the Adagrad optimizer with a general learning rate of 0.1. Convergence rates are shown in Fig. \ref{fig:results:stats} and visual examples in Fig. \ref{fig:results:engine_view} and the supplied video material. Our model is able to reproduce relatively complex motion planning such as a gradual slow down in speed before starting to rotate during a $180^o$ turn and two phase accelerations during softer turns using only 17 adjustable parameters. 

%\magnus{Convergence, increase genome count}

\changed{Fig. \ref{fig:results:stats} shows that our approach scales linear in the number of frames, has a small time foot print that allow its use in a game production environment, and that the bulk of the work is in the model optimization step, whereas trajectory estimation and adjustments are negligible.  Adagrad is a gradient descent method that modifies the general learning rate based on past steps our results show this works well for many examples, but can fail to fit.
}

\begin{figure}
    \centering
    \includegraphics[width=1.0\columnwidth]{img/stats.png}
    \caption{Left shows typical convergence rate of our local fitting optimization. Notice that $180^o$ challenges the current solver, but convergence is still achieved. On right top we show failed fittings. Lower right shows analysis of computing times as a function of frames. Observe that the overall analysis is very fast and suitable for use in game production environment.}
    \label{fig:results:stats}
\end{figure}

As more direction changes are added the error accumulates along the path.
\begin{figure}
    \centering
    \includegraphics[width=1.0\columnwidth]{img/engine_view.png}
    \caption{In-game render examples demonstrating the motion quality. Please see supplementary movie too.}
    \label{fig:results:engine_view}
\end{figure}

% \subsubsection{Automated Control Genome extraction}
% Automated genome extraction is shown in Fig. \ref{fig:results:genome_extraction}.
% \subsubsection{Alignment}
% \subsubsection{Fitting}

% \subsection{Global Fitting}

% \begin{figure}
%     \centering
%     \includegraphics[width=1.0\columnwidth]{img/locomotion mode.png}
%     \caption{}
%     \label{fig:results:locomotion_mode}
% \end{figure}



% \section{Video Notes}

% \subsubsection{Purpose}
% Illustrate the problem we wish to solve using 1-2 video examples.
% \subsubsection{Content}
% Show that basic animations have complex root motions that are difficult to replicate using standard 'weak control signal' (splines, springs). 
% Show that trivial approaches (smoothing) does not solve problem in convincing way.
% \paragraph{Videos}
% \begin{itemize}
%     \item Show character moving straight and in curves. Trace complex root motion. Highlight the presence of both high and low frequency information in the plot.
%     \item Show character running straight with smoothed trajectory. Highlight problems. Show turning characters and illustrate how superficially 'identical' turns are actually very different.
% \end{itemize}
% \paragraph{Figures}
% \begin{itemize}
%     \item Plots of trajectory and weak control signal paths.
%     \item Plots of smoothing effects
% \end{itemize}

% \subsubsection{Draft}
% Final text

% \subsection{Fitting Movement Model}
% \subsubsection{Purpose} Show that we are able to solve problem using 3-5 video examples (didactic).
% Show how various models can be fitted to data (that has itself been adjusted). 
% \begin{itemize}
%     \item Straight movement. Walk and run. 
%     \item Multiple straight movements. Walk and run. 
%     \item 180.
%     \item Multiple 180s.
%     \item Banking
%     \item Multiple banking
%     \item 45 turn
%     \item Multiple 45 turn
%     \item 90 turn
%     \item Multiple 90 turn
%     \item Mixed modes
% \end{itemize}
% Additionally we could look at animations published by ubisoft.\url{https://github.com/ubisoft/Ubisoft-LaForge-Animation-Dataset}


% \subsection{Global Fitting}

% \subsection{Application}
% \subsubsection{Purpose} Show that our solution has interesting applications using 3-5 video examples (eye candy).
% Replace standard control signal with our model.
% \begin{itemize}
%     \item{Replay animations to illustrate correct root motion}
%     \item{Use with motion matching} exponential decay = uttner, Holden Spring damper Kermse 2004].
%     \item{use generative neural network} Can we avoid floating character ?
% \end{itemize}


\section{Discussion and Limitations}
\blindtext

\section{Conclusion}
\blindtext



 \appendix
 
 \section{What you need to know?}
\blindtext

  %%
%% The acknowledgments section is defined using the "acks" environment
%% (and NOT an unnumbered section). This ensures the proper
%% identification of the section in the article metadata, and the
%% consistent spelling of the heading.
\begin{acks}
bla blah
\end{acks}
https://www.overleaf.com/project/5f605a45c274230001899537





\bibliography{movement-model-for-motion-matching}{}
\bibliographystyle{plain}
\end{document}
